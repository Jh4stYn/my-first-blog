\documentclass{article}
\usepackage[top=3cm, bottom=3cm, outer=3cm, inner=3cm]{geometry}
\usepackage{multicol}
\usepackage{graphicx}
\usepackage{url}
%\usepackage{cite}
\usepackage{hyperref}
\usepackage{array}
%\usepackage{multicol}
\newcolumntype{x}[1]{>{\centering\arraybackslash\hspace{0pt}}p{#1}}
\usepackage{natbib}
\usepackage{pdfpages}
\usepackage{multirow}
\usepackage[normalem]{ulem}
\useunder{\uline}{\ul}{}
\usepackage{svg}
\usepackage{xcolor}
\usepackage{listings}
\lstdefinestyle{ascii-tree}{
    literate={├}{|}1 {─}{--}1 {└}{+}1 
  }
\lstset{basicstyle=\ttfamily,
  showstringspaces=false,
  commentstyle=\color{red},
  keywordstyle=\color{blue}
}
%\usepackage{booktabs}
\usepackage{caption}
\usepackage{subcaption}
\usepackage{float}
\usepackage{array}

\newcolumntype{M}[1]{>{\centering\arraybackslash}m{#1}}
\newcolumntype{N}{@{}m{0pt}@{}}

%%%%%%%%%%%%%%%%%%%%%%%%%%%%%%%%%%%%%%%%%%%%%%%%%%%%%%%%%%%%%%%%%%%%%%%%%%%%
%%%%%%%%%%%%%%%%%%%%%%%%%%%%%%%%%%%%%%%%%%%%%%%%%%%%%%%%%%%%%%%%%%%%%%%%%%%%
\newcommand{\itemEmail}{jpayehuancar@unsa.edu.pe}
\newcommand{\itemStudent}{Jhastyn Jefferson Payehuanca Riquelme}
\newcommand{\itemCourse}{Programación Web 2}
\newcommand{\itemCourseCode}{20222064}
\newcommand{\itemSemester}{III}
\newcommand{\itemUniversity}{Universidad Nacional de San Agustín de Arequipa}
\newcommand{\itemFaculty}{Facultad de Ingeniería de Producción y Servicios}
\newcommand{\itemDepartment}{Departamento Académico de Ingeniería de Sistemas e Informática}
\newcommand{\itemSchool}{Escuela Profesional de Ingeniería de Sistemas}
\newcommand{\itemAcademic}{2023 - A}
\newcommand{\itemInput}{Del 08 Junio 2023}
\newcommand{\itemOutput}{Al 15 Junio 2023}
\newcommand{\itemPracticeNumber}{05}
\newcommand{\itemTheme}{Django}
%%%%%%%%%%%%%%%%%%%%%%%%%%%%%%%%%%%%%%%%%%%%%%%%%%%%%%%%%%%%%%%%%%%%%%%%%%%%
%%%%%%%%%%%%%%%%%%%%%%%%%%%%%%%%%%%%%%%%%%%%%%%%%%%%%%%%%%%%%%%%%%%%%%%%%%%%

\usepackage[english,spanish]{babel}
\usepackage[utf8]{inputenc}
\AtBeginDocument{\selectlanguage{spanish}}
\renewcommand{\figurename}{Figura}
\renewcommand{\refname}{Referencias}
\renewcommand{\tablename}{Tabla} %esto no funciona cuando se usa babel
\AtBeginDocument{%
	\renewcommand\tablename{Tabla}
}
\usepackage{fancyhdr}
\pagestyle{fancy}
\fancyhf{}
\setlength{\headheight}{30pt}
\renewcommand{\headrulewidth}{1pt}
\renewcommand{\footrulewidth}{1pt}
\fancyhead[L]{\raisebox{-0.2\height}{\includegraphics[width=3cm]{img/logo_episunsa.png}}}
\fancyhead[C]{\fontsize{7}{7}\selectfont	\itemUniversity \\ \itemFaculty \\ \itemDepartment \\ \itemSchool \\ \textbf{\itemCourse}}
\fancyhead[R]{\raisebox{-0.2\height}{\includegraphics[width=1.2cm]{img/logo_abet.png}}}
\fancyfoot[L]{Estudiante Jhastyn J.P.R.}
\fancyfoot[C]{\itemCourse}
\fancyfoot[R]{Página \thepage}

% para el codigo fuente
\usepackage{listings}
\usepackage{color, colortbl}
\definecolor{dkgreen}{rgb}{0,0.6,0}
\definecolor{gray}{rgb}{0.5,0.5,0.5}
\definecolor{mauve}{rgb}{0.58,0,0.82}
\definecolor{codebackground}{rgb}{0.95, 0.95, 0.92}
\definecolor{tablebackground}{rgb}{0.8, 0, 0}

\lstset{frame=tb,
	language=bash,
	aboveskip=3mm,
	belowskip=3mm,
	showstringspaces=false,
	columns=flexible,
	basicstyle={\small\ttfamily},
	numbers=none,
	numberstyle=\tiny\color{gray},
	keywordstyle=\color{blue},
	commentstyle=\color{dkgreen},
	stringstyle=\color{mauve},
	breaklines=true,
	breakatwhitespace=true,
	tabsize=3,
	backgroundcolor= \color{codebackground},
}

\begin{document}
	
	\vspace*{10px}
	
	\begin{center}	
		\fontsize{17}{17} \textbf{ Informe de Laboratorio \itemPracticeNumber}
	\end{center}
	\centerline{\textbf{\Large Tema: \itemTheme}}
	%\vspace*{0.5cm}	

	\begin{flushright}
		\begin{tabular}{|M{2.5cm}|N|}
			\hline 
			\rowcolor{tablebackground}
			\color{white} \textbf{Nota}  \\
			\hline 
			     \\[30pt]
			\hline 			
		\end{tabular}
	\end{flushright}	

	\begin{table}[H]
		\begin{tabular}{|x{4.7cm}|x{4.8cm}|x{4.8cm}|}
			\hline 
			\rowcolor{tablebackground}
			\color{white} \textbf{Estudiante} & \color{white}\textbf{Escuela}  & \color{white}\textbf{Asignatura}   \\
			\hline 
			{\itemStudent \par \itemEmail} & \itemSchool & {\itemCourse \par Semestre: \itemSemester \par Código: \itemCourseCode}     \\
			\hline 			
		\end{tabular}
	\end{table}		
	
	\begin{table}[H]
		\begin{tabular}{|x{4.7cm}|x{4.8cm}|x{4.8cm}|}
			\hline 
			\rowcolor{tablebackground}
			\color{white}\textbf{Laboratorio} & \color{white}\textbf{Tema}  & \color{white}\textbf{Duración}   \\
			\hline 
			\itemPracticeNumber & \itemTheme & 04 horas   \\
			\hline 
		\end{tabular}
	\end{table}
	
	\begin{table}[H]
		\begin{tabular}{|x{4.7cm}|x{4.8cm}|x{4.8cm}|}
			\hline 
			\rowcolor{tablebackground}
			\color{white}\textbf{Semestre académico} & \color{white}\textbf{Fecha de inicio}  & \color{white}\textbf{Fecha de entrega}   \\
			\hline 
			\itemAcademic & \itemInput &  \itemOutput  \\
			\hline 
		\end{tabular}
	\end{table}
	
	\section{Tarea}
	\begin{itemize}		
	\item Crea un blog sencillo en un entorno virtual utilizando la guía: https://tutorial.djangogirls.org/es/django-start-project/
		\item Especificar paso a paso la creación del blog en su informe.
		\item Crear un video tutorial donde realice las operaciones CRUD (URL public  reproducible online)
            \item  Adjuntar URL del video en el informe.
	\end{itemize}
	
	\section{URL GitHub/Video}
	\begin{itemize}
            \item URL del repositorio GitHub.
		\item \url{https://github.com/Jh4stYn/my-first-blog.git}
		\item URL del Video.
		\item \url{https://flip.com/s/iKm2W1YvNp7z}
	\end{itemize}
	
	\section{Creación del Blog}
	
	\subsection{Commits}
 
	\begin{lstlisting}[language=bash,caption={Crear una aplicación}][H]            
		$ django-admin startproject mysite .
	\end{lstlisting}	

        \lstinputlisting[language=Python, caption={Models.py},numbers=left,]{src/models.py}
 
        \begin{lstlisting}[language=bash,caption={Agregar nuestro nuevo modelo a la base de datos}][H]
		$ python manage.py makemigrations blog
      $ python manage.py migrate blog  
	\end{lstlisting}

        \lstinputlisting[language=Python, caption={Admin.py},numbers=left,]{src/admin.py}

        \begin{lstlisting}[language=bash,caption={Crear un superusuario (superuser)}][H]            
		$ python manage.py createsuperuser
        Username: Jhastyn
        Email address: jpayehuancar@unsa.edu.pe 
        Password: *********
        Password (again): *********
        Superuser created successfully.
        ----------------------------------------
	\end{lstlisting}

        \lstinputlisting[language=Python, caption={Urls.py},numbers=left,]{src/urls.py}
        
        \begin{lstlisting}[language=bash,caption={Crear urls.py en el directorio blog}][H]            
		$ vim Lab05/blog/urls.py
        from django.urls import path
        from . import views
        urlpatterns = [
            path('', views.post_list, name='post_list'),
        ]
	\end{lstlisting}
    
        \begin{lstlisting}[language=bash,caption={Agregaremos nuestras views al archivo}][H]            
    $ vim Lab05/blog/views.py
    from django.shortcuts import 
    def post_list(request):
    return render(request, 'blog/post_list.html', {})
	\end{lstlisting}

        \begin{lstlisting}[language=bash,caption={Personalizamos plantilla}][H] 
          $mkdir templates/blog
	       $vim post_list.html
            <html>
                <head>
                    <title>Django Girls blog</title>
                </head>
                <body>
                    <div>
                        <h1><a href="/">Django Girls Blog</a></h1>
                    </div>

                    <div>
                        <p>published: 14.06.2014, 12:14</p>
                        <h2><a href="">My first post</a></h2>
                        <p>Aenean eu leo quam. Pellentesque ornare sem lacinia quam venenatis vestibulum. Donec id elit non mi porta gravida at eget metus. Fusce dapibus, tellus ac cursus commodo, tortor mauris condimentum nibh, ut fermentum massa justo sit amet risus.</p>   
                    </div>

                    <div>
                        <p>published: 14.06.2014, 12:14</p>
                        <h2><a href="">My second post</a></h2>
                        <p>Aenean eu leo quam. Pellentesque ornare sem lacinia quam venenatis vestibulum. Donec id elit non mi porta gravida at eget metus. Fusce dapibus, tellus ac cursus commodo, tortor mauris condimentum nibh, ut f.</p>
                    </div>
                </body>
            </html>
        \end{lstlisting}

        \begin{lstlisting}[language=bash,caption={Abrimos views.py}][H]        
            $ vim views.py
            from django.shortcuts import render
            from django.utils import timezone
            from .models import Post

            def post_list(request):
                posts = Pst.objects.filter(published_date__lte=timezone.now()).  ...
                                                             ... order_by('published_date')
            return render(request, 'blog/post_list.html', {'posts': posts})
	\end{lstlisting}
 

        \begin{lstlisting}[language=bash,caption={Abrimos post-list.html}][H] 
            <div>
                <h1><a href="/">Django Girls Blog</a></h1>
            </div>
            
            
                <div>
                    <p> publicado: {{ post.published_date }}</p>
                    <h2><a href="">{{ post.title }}</a></h2>
                    <p>{{ post.text|linebreaksbr }}</p>
                </div>
            
        \end{lstlisting}

        \lstinputlisting[language=CSS, caption={Blog.css},numbers=left,]{src/blog.css}

        \lstinputlisting[language=html, caption={Base.html},numbers=left,]{src/base.html}
        
        \lstinputlisting[language=html, caption={Base.html},numbers=left,]{src/base.html}

        \lstinputlisting[language=html, caption={Post-list.html},numbers=left,]{src/post_list.html}

        \begin{lstlisting}[language=bash,caption={Creamos forms.py}][H]        
            $ vim forms.py
            from django import forms
            from .models import Post
            class PostForm(forms.ModelForm):
    
            class Meta:
                model = Post
                fields = ('title', 'text',)
                def post_list(request):
	\end{lstlisting}

        \begin{lstlisting}[language=bash,caption={Creamos urls.py}][H]        
            $ vim urls.py
            from django.urls import path
            from . import views
    
            urlpatterns = [
                path('', views.post_list, name='post_list'),
                path('post/<int:pk>/', views.post_detail, name='post_detail'),
                path('post/new/', views.post_new, name='post_new'),
                path('post/<int:pk>/edit/', views.post_edit, name='post_edit'),
            ]
	\end{lstlisting}

        \lstinputlisting[language=html, caption={Post-detail.html},numbers=left,]{src/post_detail.html}

        \lstinputlisting[language=html, caption={Post-edit.html},numbers=left,]{src/post_edit.html}

        \lstinputlisting[language=Python, caption={Views.py},numbers=left,]{src/views.py}
 
    \subsection{Estructura de laboratorio 05}
	\begin{itemize}	
		\item El contenido que se entrega en este laboratorio es el siguiente:
	\end{itemize}

    \begin{lstlisting}[style=ascii-tree]
    lab05/
    |--- .gitignore
    |--- manage.py
    |--- latex
    |    |--- img
    |    |   |--- logo_abet.png
    |    |   |--- logo_episunsa.png 
    |    |--- informe_lab05_Jh4stYn.pdf    
    |    |--- informe_lab05_Jh4stYn.tex
    |    |--- src
    |        |--- ListaPoo01.java
    |        |--- Node01.java
    |--- blog
    |    |--- migrations
    |    |   |--- 0001_initial.py
    |    |   |--- __init__.py 
    |    |--- static/css
    |    |   |--- blog.css
    |    |--- templates/blog
    |    |   |--- base.html
    |    |   |--- post_detail.html 
    |    |   |--- post_edit.html 
    |    |   |--- post_list.html 
    |    |--- _init_.py   
    |    |--- admin.py
    |    |--- apps.py
    |    |--- forms.py
    |    |--- models.py
    |    |--- tests.py
    |    |--- urls.py
    |    |--- views.py
    |--- myblog
    |    |--- __init__.py
    |    |--- asgi.py
    |    |--- settings.py
    |    |--- urls.py
    |    |--- wsgi.py
    |--- myenv
    |    |--- . . .
         .
         .   
         .
    \end{lstlisting}    
    
    \section{Cuestionario:¿Qué beneficios y oportunidades ofrecen las clases genéricas en Java?}
    	\begin{itemize}
    		\item ¿Cuál es un estándar de codificación para Python? \newline
     Un estándar de codificación ampliamente utilizado en la comunidad de Python es el PEP 8 (Python Enhancement Proposal 8). El PEP 8 establece recomendaciones sobre la forma en que el código Python debe estructurarse, nombrarse y formatearse para mejorar su legibilidad y consistencia.       
    		\item ¿Qué diferencias existen entre EasyInstall, pip, y PyPM?\newline
      -EasyInstall: Fue una herramienta utilizada anteriormente para instalar paquetes de Python y sus dependencias. Sin embargo, actualmente se recomienda utilizar pip en lugar de easy-install. pip proporciona más características y es ampliamente utilizado en la comunidad de Python.

     -pip: Es la herramienta de administración de paquetes estándar de Python. Se utiliza para instalar, actualizar y administrar paquetes y dependencias de Python de manera sencilla. Pip es ampliamente utilizado y cuenta con una amplia variedad de paquetes disponibles en el Python Package Index (PyPI).

     -PyPM: Es el administrador de paquetes utilizado por ActivePython, una distribución específica de Python. PyPM se utiliza para instalar paquetes y gestionar dependencias en el entorno de ActivePython. Sin embargo, ten en cuenta que PyPM no es tan ampliamente utilizado como pip, que es el administrador de paquetes estándar de Python.
    		\item En un proyecto Django que se debe ignorar para usar git.¿Qué otros tipos de archivos se deberían agregar a este archivo?\newline
      En un proyecto Django, el archivo .gitignore proporciona una lista de archivos y directorios que se deben ignorar al realizar seguimiento con Git. Algunos ejemplos de archivos y directorios comunes que se suelen agregar a .gitignore en un proyecto Django son:
    
*.pyc: Archivos de bytecode de Python generados por el intérprete.\newline
/venv/: Directorio del entorno virtual (si se utiliza).\newline
/static/: Directorio de archivos estáticos generados por Django.\newline
/media/: Directorio de archivos multimedia cargados por los usuarios.\newline
/db.sqlite3: Base de datos SQLite predeterminada de Django.\newline
/--pycache--/: Directorio de caché de Python generado por el intérprete\newline
/logs/: Directorio de registros generados por la aplicación.
                \item Utilice python manage.py shell para agregar objetos. ¿Qué archivos se modificaron al agregar más objetos?\newline
                Al utilizar python manage.py shell para agregar objetos en un proyecto Django, los archivos que se pueden modificar dependen de la forma en que se implemente la lógica de creación de objetos. Si la lógica de creación de objetos se encuentra en un archivo de vista o en un archivo de migración, esos archivos se verán modificados. Además, si se crea un objeto en una base de datos, los archivos de migración relacionados con la base de datos también pueden ser modificados para reflejar los cambios realizados.
    	\end{itemize}
    
    \end{document}
